\chapter{Introduction and Basics}
%
\section{Motivation}
bla bla  % TODO


\section{Simplicial Complexes}
Because our model fundamentally relies on \emph{simplicial complexes}
this section gives a brief review about the basic definitions and properties
used in the rest of this thesis. We include this for the sake of completeness
but \emph{not} to replace a introduction on the topic by a good text book.
You may want to consult for example (?? .. ??) % TODO
for further reading.
 
\begin{thDef}[(abstract) simplicial complex]
    \begin{itemize}
        \item
            Let $V$ be any set and let $K\subset\pot V$. The tuple $(V,K)$ is
            an \emph{(abstract) simplicial complex} if every element of~$K$ is
            a finite set, $K$ is closed under taking subsets (i.\,e. for
            $F\in K$ and $F'\subset F$ we have $F'\in K$) and $K$ contains
            all singleton subsets of~$V$ (i.\,e. for $v\in V$ we have
            $\{v\}\in K$).
            
        \item
            Let $(V,K)$ be a simplicial complex. An element~$F$ of~$K$ is
            a \emph{simplex (of~$K$)} and $\dim(F)\defeq\abs{F}-1
            \in\N\union\{-1\}$ is its dimension.
            
        \item
            The \emph{dimension of $(V,K)$} is
            \[ \dim(K) \defeq \max_{F\in K} \dim(F) \;\in\,\N\cup\{-1,\infty\}
            \]
            whenever $K\neq\emptyset$ and $-2$ otherwise.
            
        \item
            The simplicial complex $(V,K)$ is called \emph{finite} if
            $\abs{K}$ is finite.
            
        \item
            A \emph{subcomplex of $(V,K)$} is a simplicial complex~$(V',K')$
            such that $V'\subset V$ and $K'\subset K$. For $k\in\N$ the
            subcomplex
            \[ \bigl( V, \{ F\in K \Mid \dim(F) \leq k \} \bigr) \]
            of $(V,K)$ is the \emph{$k$-skeleton of $(V,K)$} which
            we denote by $(V,K)^{\leq k}$.
    \end{itemize}
\end{thDef}

\begin{thConvention}
    \begin{itemize}
        \item
            Instead of $(V,K)$ we mostly speak of a simplicial complex~$K$ where it
            is understood that $V(K)\defeq V = \bigcup K$.
            
        \item
            By an \emph{$n$-simplex} we mean a simplex of dimension~$n$.
            
        \item
            Note that every simplicial complex is partially ordered by
            inclusion and we shall occasionally use this fact without
            further notice.
    \end{itemize}
\end{thConvention}

\begin{thDef}[simplicial map]
    Let $K,L$ be simplicial complexes. A \emph{simplicial map $f\colon K\to L$}
    is a map $f\colon V(K)\to V(L)$ such that simplices of~$K$ are taken to
    simplices of~$L$, i.\,e. for $F\in K$ we have $f(F) \in L$.
\end{thDef}

\begin{thDef}[category of simplicial complexes]
    Simplicial complexes together with simplicial maps form a category~$\Simp$.
    For $n\in\N\cup\{-1\}$ we also denote its full subcategory of
    $n$-dimensional simplicial complexes by $\Simp_n$.
\end{thDef}

Analogously, we define \emph{geometric simplicial complexes}:

\begin{thDef}[geometric simplicial complex]
    Let $d\in\N$.
    \begin{itemize}
        \item
            Let $A\subset\R^d$ be a finite and affinely independet set
            of vectors. Then the convex hull
            \[ \sigma \defeq \conv(A) 
                = \bigcap \bigl\{ A'\subset\R^d 
                                \Mid A\subset A' \text{ convex} \bigr\}
            \]
            is called a \emph{(geometric) simplex (in $\R^d$)} of
            \emph{dimension $\dim(\sigma) \defeq \abs{A} - 1$}, or simply
            an \emph{$n$-simplex} if $n=\dim(\sigma)$. The elements of~$A$
            are the \emph{vertices of $\sigma$}.
            
        \item
            Let $A'\subset A$. Then $\conv(A')$ is again a simplex,
            called a \emph{face of $\sigma$}.
            %If $A'\subsetneq A$ it is a \emph{proper face}.
            
        \item
            For $n=d-1\geq 0$ the convex hull
            \[ \Delta^n \defeq \conv(e_1,\dots,e_d) \]
            of the unit vectors in $\R^d$ is the \emph{$n$-dimensional
            standard simplex}.
            
        \item
            A non-empty set $\Delta$ of geometric simplices in $\R^d$ is a
            \emph{geometric simplicial complex (in $\R^d$)} if it satisfies
            the following conditions:
            \begin{itemize}
                \item
                    For all $\sigma\in\Delta$ every face of $\sigma$ is also
                    an element of~$\Delta$.
                \item
                    For all $\sigma,\sigma'\in\Delta$ the intersection
                    $\sigma\cap\sigma'$ is a face of both $\sigma$ and
                    $\sigma'$.
            \end{itemize}
            
        \item
            Let $\Delta$ be a geometric simplicial complex in $\R^d$.
            It is \emph{finite} if $\abs{\Delta}$ is finite.
            The \emph{dimension of~$\Delta$} is
            \[ \dim(\Delta) \defeq \max_{\sigma\in\Delta} \dim(\sigma)
                \;\in\,\N\cup\{-1}
            \]
            and the \emph{polyhedron of $\Delta$} is
            \[ \norm\Delta \defeq \bigcup \Delta  \;\subset\,\R^d \]
            which we endow with the colimit topology induced by the canonical
            inclusions of the simplices.
            
        \item
            A \emph{subcomplex of $\Delta$} is a geometric simplicial
            complex~$\Delta'$ with $\Delta'\subset\Delta$.
    \end{itemize}
\end{thDef}

\begin{thExample}[standard simplex as a complex]
    For $n\in\N$ the standard simplex $\Delta^n$ together with all its faces
    forms a geometric simplicial complex. By abuse of notation we also
    write~$\Delta^n$ for this complex when the meaning is clear from the
    context.
\end{thExample}

Given a simplicial complex~$K$, we have a canonical procedure to build a
topological space from it, the so called \emph{geometric realization}:

\begin{thDef}[geometric realization]
    For a simplicial complex $K$ let
    \[ \norm{K} \defeq \coprod_{F\in K} \Delta^{\dim(F)} \,\big/\,{\sim}  , \]
    where $\sim$ is the equivalence relation that encodes the obvious gluing
    of faces that is encoded in~$K$.
    (For a more explicit description see (??).) % TODO
\end{thDef}

For a finite simplicial complex~$K$ it is easy to see, this is in fact also
(homeomorphic to) the polyhedron of a suitable geometric simplicial complex:
Let $n \defeq \abs{V(K)} - 1$ and choose a bijective map from $V(K)$ to the
set of vertices of the standard simplex~$\Delta^n$. Then the obvious subcomplex
of~$\Delta^n$ induced by this map is a geometric simplicial complex~$\Delta$
in $\R^{\abs{V(K)}}$ with $\norm{\Delta} \cong \norm{K}$.

On the other hand, for a given geometric simplicial complex, the sets of
vertices of its simplices form a finite abstract simplicial complex in the
obvious way.

\begin{thConvention}[finite complexes: abstract vs. geometric]
    When we are concerned with finite complexes, we will conveniently (but a bit
    sloppily) not strictly differentiate between abstract and geometric finite
    simplicial complexes.
    When we say \enquote{let $K$ be a (simplicial) complex}, we always mean an
    abstract complex, though.
%
%    When we say \emph{(simplicial) complex~$K$} me mostly mean an
%    abstract simplicial complex~$K$, but whenever we talk about geometric
%    (or topological) properties of~$K$ and its simplices we mean a
%    either given 
%
%    We therefore only say \emph{(simplicial) complex} by which we generelly
%    mean an abstract complex but ean either
%    type, appropriatly identified with a type of the other if necessary.
\end{thConvention}

\begin{thDef}[triangulation]
    Let $X$ be a topological space. Then $X$ is \emph{triangulable} if
    there exists a simplicial complex~$K$ and a homeomorphism
    $f\colon\norm{K}\to X$. Such a pair $(K,f)$ is then called
    a \emph{triangulation of~$X$}.
\end{thDef}


\section{Distributed Computing via Combinatorial Topology}
Now that we have an appropriate combinatorial description of (sufficiently
well-behaved) topological spaces we can define and (briefly) explain our
model for \emph{distributed tasks} and \emph{protocols that solve tasks}.
We give the strict mathematical definions first and explain their interpretation
in the context of distributed computing afterwards.

\subsection{Mathematical Framework}
The following definitions rely on an additional type of map operating on
simplicial complexes that is more flexible than simplicial maps:

\begin{thDef}[carrier map]
    Let $K,L$ be simplicial complexes. 
    \begin{itemize}
        \item
            A \emph{carrier map $\Phi\colon K\to L$} is a monotonic map
            \[ \Phi\colon K \to \{ L' \Mid L'\subset L \text{ is a subcomplex} \}  , \]
            where both sets are partially ordered by inclusion. More explicitly:
            for simplicies $F' \subset F$ in $K$ we demand $\Phi(F') \subset
            \Phi(F)$ as subcomplexes of~$L$.
            
        \item
            Let $\Phi\colon K\to L$ be a carrier map. Then $\Phi$ is
            \emph{strict} if it satisfies
            \[ \Phi(F\cap F') = \Phi(F) \cap \Phi(F') \]
            for all simplices $F,F'\in K$.
            
        \item
            For $A\subset K$ the subcomplex
            \[ \Phi[A] \defeq \bigcup \Phi(A) \]
            of~$L$ is the \emph{combined image of $A$ under $\Phi$}
            and we call $\Phi[K]$ simply \emph{combined image of~$\Phi$}.
            
        \item
            Let $\Psi\colon L\to M$ be another carrier map. Then the
            \emph{composition of $\Phi$ and $\Psi$} is defined as
            \[ (\Psi\after\Phi)(F) \defeq \Psi[\Phi(F)] \]
            for all $F\in K$ (which is a carrier map $K\to M$).
    \end{itemize}
\end{thDef}

Note: while we say \enquote{carrier map $K\to L$}, the codomain of a carrier map
is \emph{not} actually~$L$ (but the set of subcomplexes of the latter).
% FIXME: ^ maybe introduce a notation for "set of subcomplexes"!?

\begin{thDef}[carrier]
    Let $\Phi\colon K\to L$ be a strict carrier map and let $F'\in\Phi[K]$.
    The \emph{carrier of $F'$ (under $\Phi$)} is the (unique) simplex
    \[ \car_\Phi(F') \defeq \min \{ F \in K \Mid F'\in\Phi(F) \}  . \]
\end{thDef}

We can also compose carrier maps with simplicial maps in the following ways:

\begin{thDef}[composition of carrier maps with simplicial maps]
    \begin{itemize}
        \item
            For a simplicial map $f\colon K\to L$ and a subcomplex $K'$ of
            $K$ we call the subcomplex
            \[ f[K'] \defeq \{ f(F) \Mid F\in K' \} \]
            of~$L$ the \emph{image complex of $K'$ under $f$}.
            
        \item
            Now let $\Phi\colion K_2\to K_3$ be a carrier map and let
            $f\colon K_1\to K_2$ and $g\colon K_3\to K_4$ be simplicial
            maps. We define the carrier maps
            $\Phi\after f\colon K_1\to K_3$ and $g\after\Phi\colon K_2\to K_4$
            as follows:
            \[ (\Phi\after f)(F) \defeq \Phi(f(F))
                \qandq
                (g\after \Phi)(F') \defeq g[\Phi(F')]
            \]
            for all $F\in K_1$, $F'\in K_2$.
    \end{itemize}
\end{thDef}

We are now ready to define our main objects of interest, namely \emph{tasks}
and \emph{protocols} as well as the concept of \emph{task solving}.

\begin{thDef}[task and protocol]
    \label{ch1:def:taskprotocol}
    %
    \begin{itemize}
        \item
            A \emph{task} is a carrier map $\Phi\colon\cI\to\cO$.
            We call $\cI$ and $\cO$ the \emph{input} and \emph{output complex},
            respectively.
            
        \item
            A \emph{protocol} is a strict carrier map $\Xi\colon\cI\to\cP$
            where $\cP$ is the combined image of~$\Xi$.
            We call $\cI$ and $\cP$ the \emph{input} and \emph{protocol
            complex}, respectively.
    \end{itemize}
\end{thDef}

\begin{thDef}[task solving]
    Let $\cI,\cO$ and $\cP$ be simplicial complexes,
    let $\Phi\colon\cI\to\cO$ be a task,
    and let $\Xi\colon\cI\to\cP$ be a protocol.
    Then \emph{$\Xi$ solves $\Phi$} if there exists a simplicial map
    \[ \delta\colon\cP\to\cO \] 
    such that $\delta\after\Xi$ \emph{is carried by $\Phi$},
    which means that
    \[ (\delta\after\Xi)(F) \subset \Phi(F) \]
    as subcomplexes of $\cO$ is satisfied for all $F\in\cI$.
\end{thDef}

\subsection{Interpretation of the model}
It is about time that we explain some of the definitions of the previous
subsection in the context of our headline \emph{distributed computing}.
We start with an example:

\begin{thExample}[consensus]
    Let $X$ be a finite set.
    Suppose we have a system with a number of processes, each with an initial
    private \enquote{piece of information} which we require to be an element
    of~$X$. (The input values need not be distinct.) Now the processes may
    communicate (subject to certain constrains of the considered system) and
    finally have to \enquote{decide} on exactly one of the input values. That is
    all processes halt with a private output value and all of these values have
    to be the same (and additionally one of the input values). For obvious
    reasons, this task is called \emph{consensus}. Note that we are only
    interested in the set of assigned input values and the (in this example
    singleton) set of private output values.
    
    Now we explain what the corresponding input and output complexes are:
    The input complex encodes every possible \emph{input configuration},
    that is $F$ is a simplex of $\cI$ in this case if and only if
    $\emptyset\neq F\subset X$. (If it makes you more comfortable, assume that
    there are more processes than input values.) The output complex encodes
    every allowed \emph{output configuration}, that is $F'\in\cO$ if and only
    if $F'\subset X$ is a singleton set. The carrier map $\Phi$ then encodes
    which input configurations may lead to which output configurations:
    If all processes start with $x\in X$ the only allowed output configuration
    is $\{x\}$, so $\Phi(\{x\})$ would be $\{\{x\}\}$. If the input configuration
    is $\{x,y\}$ (with $x\neq y$) the processes may either choose $x$ or $y$
    as their consensus, so $\Phi(\{x,y\})$ is $\{ \{x\}, \{y\} \}$. In general:
    \[ \Phi(F) = \{ \{t\} \Mid t\in F \}  . \]
    It is readily verified that $\Phi\colon\cI\to\cO$ is a task in the sense
    of~\cref{ch0:def:taskprotocol}. Observe that $\cI$ is isomorphic to (the
    complex obtained from) the standard simplex~$\Delta^{\abs{X}-1}$ and $\cO$
    is its $0$-skeleton.
\end{thExample}

This example easily generalizes to other tasks:
$\cI$~always encodes the possible input configurations,
$\cO$~represents valid output configurations and
$\Phi\colon\cI\to\cO$ specifies the actual task, that is
for each input configuration it specifies a set of output
configurations which are considered a \enquote{successful
completion of the task} according to the task's description.

A protocol $\Xi\colon\cI\to\cP$ permitts a similar interpretation:
Again, every simplex in $\cI$ is a possible input configuration to
the system of processes. Then these processes run some sort of
algorithm whose possible output configurations are captured
by $\Xi$ and $\cP$. 
