\chapter{Introduction and Basics}
%
\section{Motivation and Notation}
Usually, \emph{topology} is considered a branch of \enquote{pure mathematics}
which has the reputation of not beeing particularly usefull in other fields than
mathematics itself. In this thesis, however, we use topological methods for
a theoretical model of \emph{distributed computing}, especially so called
\emph{loop agreement tasks}. Distributed computing is the study of writing
programs that run simultaneously on multiple proccessing units, which
communicate through a given network protocol. Often it is a complex problem to
decide whether a given task can be solved under certain preconditions depending
on the hardware environment and the constrains imposed by the communication
protocol. To answer this question for some cases, we define and employ an
abstract model of distributed computing in \cref{ch2}, using \emph{simplicial
complexes} and so called \emph{carrier maps} (which are a generalisation of
\emph{simplicial maps}). Here we follow 
Herlihy et~al.~\cite{bookc:herlihyetal13} and adopt their definitions found
in Chapter~4 of their book. Then we are going to describe \emph{loop agreement
tasks} and establish a link to the \emph{fundamental group of a space}. The
corresponding classification result of \cref{ch3} is taken from
\cite[Chapter~15]{bookc:herlihyetal13}, but our approach and proof will be a bit
different. (If you do not have the book available, most of the chapter can also
be found in an earlier paper by Herlihy and
Rajsbaum~\cite{paper:herlihyrajsbaum03}.) In the last chapter we investigate how
the preceeding material can be applied to loop agreement tasks living on
\emph{(two-dimensional) pseudomanifolds}, accompanied by an implementation in
\emph{Haskell}.

The tasks and protocols (i.\,e. distributed programs) captured by our simplicial
complex model are subject to the following real world conditions and properties:
Most importantly, the processes run \emph{asynchronously}, that is, a process
cannot wait for another process. It has to do its computation in a finite number
of steps with whatever information it got up to that point. Therefore, such
programs are called \emph{wait-free}. This also includes the problem of
\enquote{crashed processes} (processes that halted prematurely, e.\,g. because
of a hardware malfunction), because in a wait-free computation a process cannot
know how many other processes are still running. Secondly, tasks are
\emph{colorless}, which means that it is irrelevant which process has which
input (respectively output) value, only the set of input (respectively output)
values is considered. And lastly, the communication between processes happens
through \emph{shared memory}, every outgoing communication of a process consists
of \enquote{everything it knows} and reading the memory always happens as an
\emph{atomic snapshot}. The method of communication does not actually affect
our model, but Herlihy et~al.~\cite{bookc:herlihyetal13} develop their theory of
\emph{colorless layered immediate snapshot protocols} and protocols that can be
built from those around the above assumptions, so we include the latter to
justify our use of so called \emph{barycentric agreement protocols} in the
following.

We get back to the model and its interpretation in \cref{ch2}, but we will not
go into much more detail concerning the above technical terms. More explanations
on those terms in the context of distributed computing and how they relate to
the mathematical model can be found in the book by
Herlihy et~al.~\cite{bookc:herlihyetal13}, a paper by
Herlihy and Shavit~\cite{paper:herlihyshavit99}, and a paper by
Herlihy and Rajsbaum~\cite{paper:herlihyrajsbaum03}.

\bigskip
Throughout this thesis, we use the following notation and conventions:
\begin{itemize}
    \item
        We write $\subset$ for \enquote{subset or equal}
        and $\subsetneq$ for \enquote{proper subset}.
        
    \item
        For a set of sets $A$, the symbol $\bigcup A$ denotes the
        union over all elements of~$A$, also commonly written as
        $\bigcup_{B\in A} B$.
        
    \item
        The natural numbers $\N$ start at $0$.
        
    \item
        A \enquote{map} out of a topological space always means
        \enquote{continuous map}, unless stated otherwise.
\end{itemize}

\section{Simplicial Complexes}
Since our mathematical model for distributed computing fundamentally relies on
\emph{simplicial complexes}, the following contains a brief review of the basic
definitions and properties used in this thesis. We include it for the sake of
completeness but \emph{not} to replace an introduction on the topic by a good
text book. You may want to consult, for example,
Munkres~\cite[\S1\,ff. and \S14\,ff.]{bookc:munkres84},
Fritsch and Piccinini~\cite[Ch.~3]{bookc:fritschpiccinini90}
or Spanier~\cite[Ch.~3]{bookc:spanier66}
for further reading.
 
\begin{thDef}[(abstract) simplicial complex]\hfill
    \begin{itemize}
        \item
            Let $V$ be a set and let $K$ be a subset of the powerset of~$V$.
            The pair $(V,K)$ is an \emph{(abstract) simplicial complex} if every
            element of~$K$ is a finite set, $K$ is closed under taking subsets
            (i.\,e. for all $F\in K$ and $F'\subset F$ we have $F'\in K$) and
            $K$ contains all singleton subsets of~$V$ (i.\,e. for all $v\in V$
            we have $\{v\}\in K$).
            
        \item
            Let $(V,K)$ be a simplicial complex. An element~$v$ of~$V$ is
            a \emph{vertex of~$K$}. An element~$F$ of~$K$ is
            a \emph{simplex (of~$K$)}, $\dim(F)\defeq\abs{F}-1
            \in\N\cup\{-1\}$ is its \emph{dimension}, and each subset of~$F$ is
            called a \emph{face of~$F$}. A simplex of dimension~$n$ is also
            called an \emph{$n$-simplex}.
            
        \item
            The \emph{dimension of $(V,K)$} is
            \[ \dim(K) \defeq \max_{F\in K} \,\dim(F) \;\in\,\N\cup\{-1,\infty\}
            \]
            whenever $K\neq\emptyset$ and $-2$ otherwise.
            
        \item
            The simplicial complex $(V,K)$ is called \emph{finite} if
            $\abs{K}$ is finite.
            
        \item
            A \emph{subcomplex of $(V,K)$} is a simplicial complex~$(V',K')$
            such that $V'\subset V$ and $K'\subset K$. For $k\in\N$ the
            subcomplex
            \[ \bigl( V,\; \{ F\in K \cMid[\,]{} \dim(F) \leq k \} \bigr) \]
            of $(V,K)$ is the \emph{$k$-skeleton of $(V,K)$}, which
            we denote by $(V,K)^{\leq k}$.
            
        \item
            Let $F$ be a simplex of~$K$. Then $K(F)$ denotes the subcomplex
            of~$(V,K)$ determined by $F$ and its faces, i.\,e.
            $K(F) = (F, \, \{ F' \Mid F'\subset F \})$.
    \end{itemize}
\end{thDef}

\pagebreak[2]
\begin{thConvention}\hfill
    \begin{itemize}
        \item
            Instead of $(V,K)$ we mostly speak of a simplicial complex~$K$ where it
            is understood that $V(K)\defeq V = \bigcup K$.
            
        \item
            Note that every simplicial complex is partially ordered by
            inclusion and we shall occasionally use this fact without
            further notice.
    \end{itemize}
\end{thConvention}

\begin{thDef}[simplicial map]
    Let $K,L$ be simplicial complexes. A \emph{simplicial map $f\colon K\to L$}
    is a map $f\colon V(K)\to V(L)$ such that simplices of~$K$ are taken to
    simplices of~$L$, i.\,e. for $F\in K$ we have $f(F) \in L$.
\end{thDef}

\begin{thDef}[category of simplicial complexes]
    Simplicial complexes together with simplicial maps (and usual function
    composition) form a category~$\Simp$.
    For $n\in\N\cup\{-1\}$ we also denote its full subcategory of
    $n$-dimensional simplicial complexes by $\Simp_n$.
\end{thDef}

\begin{thExample}[standard simplex (as a complex)]
    Let $n\in\N$. The \emph{$n$-dimensional standard simplex} is
    \[ \Delta^n \defeq \setZeroto n  . \]
    The standard simplex~$\Delta^n$ together with all its faces forms a
    simplicial complex, which (by abuse of notation) we also denote by~$\Delta^n$
    (the meaning will be clear from the context). We write $\partial\Delta^n$
    for the $(n{-}1)$-skeleton of $\Delta^n$.
\end{thExample}

\begin{thDef}[geometric standard simplex]
    Let $n\in\N$. The \emph{$n$-dimensional geometric standard simplex} is
    the convex hull of the unit vectors in~$\R^{n+1}$ (with the subspace
    topology):
    \[ \Deltageo^n \defeq \conv(e_0,\dots,e_{n}) \;\subset\,\R^{n+1} . \]
    For $J\subset\setZeroto n$ the subspace $\conv(\{e_j\Mid j\in J\})$
    of~$\Deltageo^n$ is called a \emph{face of~$\Deltageo^n$}.
\end{thDef}

\begin{thDef}[geometric realization]\hfill
    \begin{itemize}
        \item
            Let $K$ be a simplicial complex. The \emph{geometric realization
            of~$K$} is the topological space
            \[ \geom{K} \defeq \coprod_{F\in K} \Deltageo^{\dim(F)}
                \,\big/\,{\sim}
            , \]
            where $\sim$ is the equivalence relation generated by the obvious
            gluing of faces that is encoded in~$K$.
            For a vertex~$v$ and a simplex~$F$ of~$K$ we denote by $\geom v$ 
            and~$\geom F$ the corresponding point and subspace of~$\geom K$,
            respectively (i.\,e. the image (point) under the canonical inclusion
            followed by the quotient map). Consequently, for a subspace
            $K'\subset K$ we write $\geom{K'}$ for $\bigcup \{ \geom{F'} \Mid
            F'\in K' \} \subset \geom K$. For $x\in\geom K$ the (unique)
            simplex~$F$ of~$K$ such that the interior of~$\geom F$ contains~$x$
            is $\supp(x) = F$.
            
        \item
            Let $K,L$ be simplicial complexes and let $f\colon K\to L$ be a
            simplicial map. Then
            \[ \geom{f}\colon\geom{K}\to\geom{L} \]
            is the continuous map obtained from~$f$ by affine extension to each
            simplex.
    \end{itemize}
\end{thDef}
%
More explicit definitions and explanations of geometric realizations can be
found in the references mentioned at the beginning of this section.

\pagebreak[2]
\begin{thDef}[barycentric subdivision]
    Let $K$ be a simplicial complex. The \emph{(first) barycentric subdivsion
    of~$K$} is the simplicial complex
    \[ \sd K  \defeq \bigl\{
            \{ F_1,\dots,F_k \} \subset K \Mid
            k\in\N, \; F_1\subsetneq\dots\subsetneq F_k
        \bigr\}
    . \]
    We set $\sd^0 K \defeq K$, $\sd^1 K \defeq \sd K$ and for $N\in\N[\geq2]$ we
    define recursively $\sdNK \defeq \sd(\sd^{N-1} K)$, the \emph{$N$-th
    barycentric subdivision of~$K$}.
\end{thDef}

\begin{thProposition}[geometric barycentric subdivision]
    Let $K$ be a simplicial complex. 
    \begin{itemize}
        \item
            There exists a homeomorphism
            \[ \beta_K\colon \geom{\sd K} \to \geom{K}  , \]
            given by the affine extension to each simplex of the following map
            $\geom{(\sd K)^{\leq 0}}\to\geom{K}$: for a vertex~$v$ of $\sd K$,
            which is a simplex of~$K$, say $v = F\in K$, the point
            $\geom{v}\in\geom{\sd K}$ is mapped to the barycenter of
            $\geom F\subset\geom K$.
            
        \item
            In particular, the above homeomorphism induces a homeomorphism
            \[ \beta_K^N\colon\geom{\sdNK}\to\geom{K} \]
            for all $N\in\N[\geq1]$, and for convenience, we set
            $\beta_K^0 \defeq \id_{\geom K}$.
    \end{itemize}
\end{thProposition}

A proof can be found in the book by Fritsch and
Piccinini~\cite[Proposition~3.3.16]{bookc:fritschpiccinini90}.

\begin{thDef}[simplicial approximation]
    Let $K,L$ be simplicial complexes and let $f_\geo\colon\geom K\to\geom L$
    be continuous.  A simplicial map $f\colon K\to L$ is a \emph{simplicial
    approximation to~$f_\geo$} if $\supp(\geom{f}(x))$ is a face of
    $\supp(f_\geo(x))$ for all $x\in\geom K$.
\end{thDef}

\begin{thTheorem}[simplicial approximation theorem]
    \label{ch1:simpapprox}
    %
    Let $K,L$ be finite simplicial complexes and let
    $f_\geo\colon\geom K\to\geom L$ be continuous.
    Then there exists an $N\in\N$ and a simplicial map $f\colon\sdNK\to L$
    such that $f$ is a simplicial approximation to
    $f_\geo\after\beta_K^N\colon\geom{\sdNK}\to\geom L$.
\end{thTheorem}
%
See, for example,
Munkres~\cite[Theorem~16.1]{bookc:munkres84}
or Bredon~\cite[Theorem~22.10]{bookc:bredon93}.

\begin{thLemma}[inclusion of subcomplex is cofibration]
    \label{ch1:inclcofibration}
    %
    Let $K$ be a simplicial complex and let $K'$ be a subcomplex
    of~$K$. Then $\geom{K'}\hookrightarrow\geom{K}$ is a cofibration.
\end{thLemma}
%
See, for example, Spanier~\cite[Ch.~3,\;Sec.~2,\;Corollary~5]{bookc:spanier66}.
