\chapter{Introduction and Basics}
%
\section{Motivation}
bla bla  % TODO


\section{Simplicial Complexes}
Because our model fundamentally relies on \emph{simplicial complexes}
this section gives a brief review about the basic definitions and properties
used in the rest of this thesis. We include this for the sake of completeness
but \emph{not} to replace a introduction on the topic by a good text book.
You may want to consult for example (?? .. ??) % TODO
for further reading.
 
\begin{thDef}[(abstract) simplicial complex]\hfill
    \begin{itemize}
        \item
            Let $V$ be any set and let $K\subset\pot V$. The tuple $(V,K)$ is
            an \emph{(abstract) simplicial complex} if every element of~$K$ is
            a finite set, $K$ is closed under taking subsets (i.\,e. for
            $F\in K$ and $F'\subset F$ we have $F'\in K$) and $K$ contains
            all singleton subsets of~$V$ (i.\,e. for $v\in V$ we have
            $\{v\}\in K$).
            
        \item
            Let $(V,K)$ be a simplicial complex. An element~$F$ of~$K$ is
            a \emph{simplex (of~$K$)}, $\dim(F)\defeq\abs{F}-1
            \in\N\cup\{-1\}$ is its dimension, and each subset of~$F$ is
            called a \emph{face of~$F$}.
            
        \item
            The \emph{dimension of $(V,K)$} is
            \[ \dim(K) \defeq \max_{F\in K} \,\dim(F) \;\in\,\N\cup\{-1,\infty\}
            \]
            whenever $K\neq\emptyset$ and $-2$ otherwise.
            
        \item
            The simplicial complex $(V,K)$ is called \emph{finite} if
            $\abs{K}$ is finite.
            
        \item
            A \emph{subcomplex of $(V,K)$} is a simplicial complex~$(V',K')$
            such that $V'\subset V$ and $K'\subset K$. For $k\in\N$ the
            subcomplex
            \[ \bigl( V,\; \{ F\in K \cMid[\,]{} \dim(F) \leq k \} \bigr) \]
            of $(V,K)$ is the \emph{$k$-skeleton of $(V,K)$} which
            we denote by $(V,K)^{\leq k}$.
            
        \item
            Let $F$ be a simplex of~$K$. Then $K(F)$ denotes the subcomplex
            of~$(V,K)$ determined by $F$ and its faces, i.\,e.
            $K(F) = (F, \, \{ F' \Mid F'\subset F \})$.
    \end{itemize}
\end{thDef}

\begin{thConvention}\hfill
    \begin{itemize}
        \item
            Instead of $(V,K)$ we mostly speak of a simplicial complex~$K$ where it
            is understood that $V(K)\defeq V = \bigcup K$.
            
        \item
            By an \emph{$n$-simplex} we mean a simplex of dimension~$n$.
            
        \item
            Note that every simplicial complex is partially ordered by
            inclusion and we shall occasionally use this fact without
            further notice.
    \end{itemize}
\end{thConvention}

\begin{thDef}[simplicial map]
    Let $K,L$ be simplicial complexes. A \emph{simplicial map $f\colon K\to L$}
    is a map $f\colon V(K)\to V(L)$ such that simplices of~$K$ are taken to
    simplices of~$L$, i.\,e. for $F\in K$ we have $f(F) \in L$.
\end{thDef}

\begin{thDef}[category of simplicial complexes]
    Simplicial complexes together with simplicial maps form a category~$\Simp$.
    For $n\in\N\cup\{-1\}$ we also denote its full subcategory of
    $n$-dimensional simplicial complexes by $\Simp_n$.
\end{thDef}

Analogously, we define \emph{geometric simplicial complexes}:

\begin{thDef}[geometric simplicial complex]
    Let $d\in\N$.
    \begin{itemize}
        \item
            Let $A\subset\R^d$ be a finite and affinely independet set
            of vectors. Then the convex hull
            \[ \sigma \defeq \conv(A) 
                =\mkern1mu \bigcap \mkern1mu \bigl\{ A'\subset\R^d 
                                \Mid A\subset A' \text{ convex} \bigr\}
            \]
            is called a \emph{(geometric) simplex (in $\R^d$)} of
            \emph{dimension $\dim(\sigma) \defeq \abs{A} - 1$}, or simply
            an \emph{$n$-simplex} if $n=\dim(\sigma)$. The elements of~$A$
            are the \emph{vertices of $\sigma$}.
            
        \item
            Let $A'\subset A$. Then $\conv(A')$ is again a simplex,
            called a \emph{face of $\sigma$}.
            %If $A'\subsetneq A$ it is a \emph{proper face}.
            
        \item
            For $n=d-1\geq 0$ the convex hull
            \[ \Delta^n \defeq \conv(e_1,\dots,e_d) \]
            of the unit vectors in $\R^d$ is the \emph{$n$-dimensional
            standard simplex}.
            
        \item
            A set $\Delta$ of geometric simplices in $\R^d$ is a
            \emph{geometric simplicial complex (in~$\R^d$)} if it satisfies
            the following conditions:
            \begin{itemize}[topsep=0pt]
                \item
                    For all $\sigma\in\Delta$ every face of $\sigma$ is also
                    an element of~$\Delta$.
                \item
                    For all $\sigma,\sigma'\in\Delta$ the intersection
                    $\sigma\cap\sigma'$ is a face of both $\sigma$ and
                    $\sigma'$.
            \end{itemize}
            \smallskip
            
        \item
            Let $\Delta$ be a geometric simplicial complex in $\R^d$.
            It is \emph{finite} if $\abs{\Delta}$ is finite.
            The \emph{dimension of~$\Delta$} is
            \[ \dim(\Delta) \defeq \max_{\sigma\in\Delta} \, \dim(\sigma)
                \;\in\,\N\cup\{-1\}
            \]
            and the \emph{polyhedron of $\Delta$} is
            \[ \geom\Delta \defeq \bigcup \Delta  \;\subset\,\R^d \]
            which we endow with the colimit topology induced by the canonical
            inclusions of the simplices.
            
        \item
            A \emph{subcomplex of $\Delta$} is a geometric simplicial
            complex~$\Delta'$ with $\Delta'\subset\Delta$.
    \end{itemize}
\end{thDef}

\begin{thExample}[standard simplex as a complex]
    For $n\in\N$ the standard simplex $\Delta^n$ together with all its faces
    forms a geometric simplicial complex. By abuse of notation we also
    write~$\Delta^n$ for this complex when the meaning is clear from the
    context.
\end{thExample}

\begin{thDef}[geometric realization]\hfill
    \begin{itemize}
        \item
            For a simplicial complex $K$ let
            \[ \geom{K} \defeq \coprod_{F\in K} \Delta^{\dim(F)} \,\big/\,{\sim}  , \]
            where $\sim$ is the equivalence relation generated by the obvious gluing
            of faces that is encoded in~$K$.
            The topological space $\geom{K}$ is the \emph{geometric realization} of~$K$.
            
        \item
            Let $f\colon K\to L$ be a simplicial map. Then
            $\geom{f}\colon\geom{K}\to\geom{L}$ is the continuous map
            obtained from~$f$ by affine extension to each simplex.
    \end{itemize}
\end{thDef}
%
For more explicit definitions see (?? .. ??). % TODO

\begin{thDef}[triangulation]
    Let $X$ be a topological space. Then $X$ is \emph{triangulable} if
    there exists a simplicial complex~$K$ and a homeomorphism
    $f\colon\geom{K}\to X$. Such a pair $(K,f)$ is then called
    a \emph{triangulation of~$X$}.
\end{thDef}

\begin{thDef}[barycentric subdivision]
    Let $K$ be a simplicial complex. The \emph{(first) barycentric subdivsion
    of~$K$} is the simplicial complex
    \[ \sd K  \defeq \bigl\{
            \{ F_1,\dots,F_k \} \subset K \Mid
            k\in\N, \; F_1\subsetneq\dots\subsetneq F_k
        \bigr\}
    . \]
    We set $\sd^0 K \defeq \sd K$ and for $N\in\N[\geq1]$ we
    define recursively $\sd^N K \defeq \sd(\sd^{N-1} K)$,
    the \emph{$N$-th barycentric subdivision of~$K$}.
\end{thDef}

% TODO v
%\begin{thDef}[simplicial approximation]
%
%\end{thDef}
%
%\begin{thTheorem}[simplicial approximation theorem]
%
%\end{thTheorem}


