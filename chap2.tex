\chapter{Loop Agreement Tasks}
\section{Definition and Examples}
\emph{Loop agreement tasks} form a class of tasks in the sense of
\cref{ch1:taskprotocol}. They are related to loops in topolgical spaces
and in fact our main result in this chapter \pcref{.} % TODO
connects the sovability of one such task through another with the
\emph{fundamental groups} of the (geometric realizations of the)
involved complexes. We start with the definition of a combinatorical
model of loops:

\begin{thDef}[walk, path, cycle]
    Let $K$ be a simplicial complex.
    \begin{itemize}
        \item
            A \emph{walk in $K$} is a tuple $(v_0,\dots,v_n)$
            where $n\in\N$ and $v_0,\dots,v_n$ are vertices
            of~$K$ such that $\{v_j,v_{j+1}\}$ is a $1$-simplex of~$K$
            for all $j\in\setZeroto{n-1}$.
            
        \item
            A \emph{(simple) path in $K$} is a walk in~$K$ such that the
            vertices of the walk are pairwise distinct.
            
        \item
            A \emph{cycle in $K$} is a walk $(v_0,\dots,v_n)$ such that
            $n\in\N[\geq2]$, $v_0 = v_n$ and $(v_0,\dots,v_{n-1})$ is a path.
    \end{itemize}
\end{thDef}

Our terminology concides with the typical definitions found in an introduction
to graph theory and as a matter of fact the above definition is just a
restatement of the classical terms in the context of the graph~$K^{\leq 1}$.


\section{Classification}

\section{Consequences}
\label{ch2:sec:consequences}
