\chapter{Loop Agreement Tasks}
\label{ch3}
\section{Definition and Examples}
\emph{Loop agreement tasks} form a class of tasks in the sense of
\cref{ch2:def:taskprotocol}. They are related to loops in topological spaces
and in fact our main result in this chapter \pcref{ch3:classification}
connects the solvability of one such task through another with the
\emph{fundamental groups} of (the geometric realizations of) the
involved complexes. We start with the definition of a combinatorial
model of loops:

\begin{thDef}[walk, path, cycle]
    \label{ch3:def:walkpathcycle}
    %
    Let $K$ be a simplicial complex.
    \begin{itemize}
        \item
            A \emph{walk in $K$} is a tuple $w = (v_0,\dots,v_n)$
            where $n\in\N$ and $v_0,\dots,v_n$ are vertices
            of~$K$ such that $\{v_j,v_{j+1}\}$ is a $1$-simplex of~$K$
            (called an \emph{edge of $w$}) for all $j\in\setZeroto{n-1}$.
            If $w$ is a walk in~$K$ as above, we denote by $\dot w \defeq v_0$
            and $\ddot w \defeq v_n$ the first and last vertex of~$w$,
            respectively.
            
        \item
            A \emph{(simple) path in $K$} is a walk in~$K$ such that its
            vertices are pairwise distinct.
            
        \item
            A \emph{cycle in $K$} is a walk $(v_0,\dots,v_n)$ in~$K$ such that
            $n\in\N[\geq2]$, $v_0 = v_n$ and $(v_0,\dots,v_{n-1})$ is a path.
    \end{itemize}
\end{thDef}

In the context of the graph~$K^{\leq 1}$ our terminology coincides with the
typical definitions found in an introduction to graph theory and, indeed, the
above definition is just a restatement of the classical terms in the simplicial
setting.

\begin{thDef}[composition of walks]
    Let $K$ be a simplicial complex, let $k\in\N[\geq1]$ and let $w_1,\dots,w_k$
    be walks in~$K$ where $w_j = (v_{j,0}, \dots, v_{j,n_j})$ for all
    $j\in\setOneto{k}$.  We say $w_1\ast\dots\ast w_k$ is a \emph{valid
    composition} if $\ddot{w}_j = \dot{w}_{j+1}$ for all $j\in\setOneto{k-1}$
    and in that case we define the \emph{composition of $w_1,\dots,w_k$} as the
    walk
    \[ w_1\ast\dots\ast w_k
        \defeq (v_{1,0},\dots,v_{1,n_1-1},
                \dots,
                v_{k-1,0},\dots,v_{k-1,n_{k-1}-1},
                v_{k,0},\dots,v_{k,n_k})
    . \]
\end{thDef}

\begin{thExample}
    On the torus complex~$T$ from \cref{ch2:fig:carriermaps} the following
    holds:
    \begin{itemize}
        \item
            $w \defeq (1,3,9)$ is a path (but not a cycle) with
            $\dot w = 1$ and $\ddot w = 9$.
        \item
            $(3,5,9,5,3)$ is a (closed) walk but neither a path nor a cycle
            (because the vertices $5$ and $3$ appear twice).
        \item
            $(3,5,9,1,3)$ is a cycle.
\pagebreak[3] % !!! layout hack
        \item
            $(1,2) \ast (2,3,1) = (1,2,3,1)$.
        \item
            $(2,5,1,2)$ is not even a walk because $\{2,5\}$ is not a simplex
            of~$T$.
    \end{itemize}
\end{thExample}

\pagebreak[2]
\begin{thDef}[triangle loop]
    Let $K$ be a simplicial complex.
    A \emph{triangle loop in $K$} is a triple $\kappa = (p_0,p_1,p_2)$ of paths
    in~$K$ such that $p_0\ast p_1\ast p_2$ is a valid composition and a cycle
    in~$K$. In that case we call $\dot{p}_0,\dot{p}_1,\dot{p}_2$ the
    \emph{distinguished vertices} and $\dot\kappa \defeq \dot{p}_0$ the
    \emph{base point} of~$\kappa$.
\end{thDef}

Given a triangle loop in~$K$, this loop gives rise to a distributed task:
the processes are assigned input values that correspond to the distinguished
vertices of the loop and the output configuration has to be a simplex of~$K$
that \enquote{lies (on the path) between the input vertices}. We shall make this
precise soon, but first we introduce some useful conventions.

\begin{thConvention}\hfill
    \begin{itemize}
        \item
            Let $K$ be a simplicial complex and let $\ppp$ be a triangle loop
            in~$K$. For $j\in\{0,1,2\}$ the subcomplex of~$K$ determined
            by~$p_j$ is denoted $K(p_j)$; more precisely: $K(p_j)$ is the
            smallest subcomplex of~$K$ containing all edges of~$p_j$.
            
        \item
            Let $\finSimp$ denote the full subcategory of $\Simp_2$ that has
            \emph{finite connected $2$-dimensional simplicial complexes} as its
            objects. A simplicial complex~$K$ is \emph{connected} if for
            every two vertices $v,v'\in V(K)$ there exists a walk~$w$ in~$K$
            with $\dot w = v$ and $\ddot w = v'$. (This clearly implies that
            $\geom K$ is path connected in the usual topological sense.)
    \end{itemize}
\end{thConvention}

\begin{thDef}[loop agreement task]
    Let $K\in\finSimp$ and let $\kappa = \ppp$ be a triangle loop in~$K$. The
    associated \emph{loop agreement task $\Loop{K,\kappa}\colon\Delta^2 \to K$}
    is defined as:
    \[
        F \mapsto \begin{cases}
            K(\{\dot{p}_j\}) &\quad \text{if } F = \{j\},   \\
            K(p_0)           &\quad \text{if } F = \{0,1\}, \\
            K(p_1)           &\quad \text{if } F = \{1,2\}, \\
            K(p_2)           &\quad \text{if } F = \{0,2\}, \\
            K             &\quad \text{if } F = \{0,1,2\}
        . \end{cases}
    \]
\end{thDef}

Our informal description above has an exact interpretation now: if the input
configuration is, say, $\{0,1\}\in\Delta^2$, the output configuration has to
be a simplex of $K(p_0)\subset K$, that is a vertex or edge of the given
path~$p_0$ in~$K$.

\begin{thExample}
    The example \ref{ch2:ex:carriermaps:lat} of \cref{ch2:ex:carriermaps}
    actually describes a loop agreement task: the carrier map~$\Phi_3$
    coincides with $\Loop{T,\tau}$ for $\tau = ((3,5),(5,9,1),(1,3))$.
\end{thExample}

\pagebreak[2]
\begin{thExample}[$(3,2)$-set agreement as loop agreement task]
    \label{ch3:32setagreement}
    %
    As a first but important example, we observe that the
    $(3,2)$-set agreement task \pcref{ch2:setagreement}
    \[ \skel^1\colon\Delta^2\to\partial\Delta^2 \]
    is the same as the loop agreement task
    \[ \Loop{\partial\Delta^2,\,\kappa}
        \qtextq{for}
        \kappa = \bigl( (0,1),\, (1,2),\, (2,0) \bigr)
    . \]
\end{thExample}

\begin{thExample}[$2$-dimensional barycentric agreement as loop agreement task]
    \label{ch3:2dimbaryagreement}
    %
    Let $N\in\N$. Let $p_0$ be the unique path from $\sd_{\Delta^2}^N(\{0\})$
    to $\sd_{\Delta^2}^N(\{1\})$ in $\sd_{\Delta^2}^N(\{0,1\})$ and let
    $p_1,p_2$ be defined analogously. Then the barycentric agreement task
    \[ \sd_{\Delta^2}^N\colon \Delta^2 \to \sd^N \Delta^2 \]
    is the same as the loop agreement task
    \[ \Loop{\sd^N\!\Delta^2, \; (p_0,p_1,p_2)}  . \]
\end{thExample}

Some more examples can be found in the book by
Herlihy et~al.~\cite[Sec.~5.6.3]{bookc:herlihyetal13}.

\section{Topological Aspects}
In this section we define an \emph{algebraic signature} of a loop agreement task
based on the \emph{fundamental group} of its simplicial complex's geometric
realization. We also show that every homomorphism of fundamental groups where
the domain space is the geometric realization of a finite, $2$-dimensional
simplicial complex is in fact induced by a continuous map and how that affects
\emph{maps between algebraic signatures}.

\begin{thDef}[topological loop associated to a triangle loop]\hfill
    \label{ch3:def:assoctopoloop}
    %
    Let $K\in\finSimp$, let $c=(v_0,\dots,v_n)$ be a cycle in~$K$,
    and let $\kappa=\ppp$ be a triangle loop in~$K$.
    \begin{itemize}
        \item
            For $j\in\setZeroto{n-1}$ let $i_j\colon\geom{\{v_j,v_{j+1}\}}
            \hookrightarrow\geom K$ be the canonical inclusion.
            The \emph{(topological) loop~$\gamma_c$ associated to~$c$} is the
            (well-defined) injective continuous map
            \[
                S^1\to\geom K, \quad
                [t]\mapsto i_j(1-\tau,\tau) \qtextq{for} \tau=tn-j,
                    \qtextq{if} t\in\bigl[\tfrac{j}{n}, \tfrac{j+1}{n}\bigr]
            . \]
            
        \item
            The \emph{(topological) loop~$\gamma_\kappa$ associated to~$\kappa$}
            is
            \[ \gamma_\kappa \defeq \gamma_{p_0\ast p_1\ast p_2}  . \]
    \end{itemize}
\end{thDef}

\begin{thConvention}[fundamental group of a complex]
    Let $K$ be a simplicial complex and let $v$ be a vertex of~$K$. We say
    \emph{fundamental group of~$K$ (based at $v$)} for the \emph{fundamental
    group of~$\geom K$ (based at~$\geom v$)} and write
    \[ \pi_1(K,v) \defeq \pi_1(\geom K, \geom v) . \]
\end{thConvention}

\begin{thDef}[algebraic signature]
    Let $K,L\in\finSimp$ and let $\kappa,\lambda$ be triangle loops in
    $K$ and $L$, respectively.
    \begin{itemize}
        \item
            The \emph{algebraic signature %
                      of the loop agreement task~$\Loop{K,\kappa}$}
            is the pair
            \[ \bigl( \pi_1(K,\dot\kappa), \, [\gamma_\kappa]_\ast \bigr) \]
            where the second element is the pointed homotopy class
            of~$\gamma_\kappa$ as an element of $\pi_1(K,\dot\kappa)$.
            
        \item
            If $\phi\colon\pi_1(K,\dot\kappa)\to\pi_1(L,\dot\lambda)$ is a
            group homomorphism that maps $[\gamma_\kappa]_\ast$ to
            $[\gamma_\lambda]_\ast$, we call
            \[ \phi\colon \bigl( \pi_1(K,\dot\kappa), \, [\gamma_\kappa]_\ast \bigr)
                \to \bigl( \pi_1(L,\dot\lambda), \, [\gamma_\lambda]_\ast \bigr)
            \]
            a \emph{map of algebraic signatures}.
    \end{itemize}
\end{thDef}

In general, a group homomorphism between fundamental groups of topological spaces
need not be induced by a continuous map. For finite simplicial complexes,
however, we have the following

\begin{thLemma}[continuous realization of homomorphism]
    \label{ch3:continuousrealization}
    %
    Let $K$ be a finite connected simplicial complex with $\dim(K)\leq 2$, let
    $x$ be a vertex of $K$, let $(Y,y_0)$ be a pointed topological space, and
    let $\phi\colon\pi_1(K,x)\to\pi_1(Y,y_0)$ be a group homomorphism.
    Then there exists a base point preserving continuous map
    $f\colon(\geom K,\geom x)\to(Y,y_0)$ such that $\pi_1(f) = \phi$.
\end{thLemma}

\begin{proof}
    Let $K^\ast$ be a spanning tree of the graph~$K^{\leq 1}$ and let
    $x_0\defeq\geom x\in\geom K$. Then $\geom{K^\ast} \subset \geom K$ is
    contractible and since the inclusion of a subcomplex is a cofibration
    \pcref{ch1:inclcofibration} this implies that the quotient map $q\colon
    \geom K \to \geom K / \geom{K^\ast} \eqdef X$ is a homotopy equivalence.
    Let
    \[ \psi \defeq \phi\after\pi_1(q)^{-1}\colon
        \pi_1(X,[x_0]) \to \pi_1(Y,y_0)
    \]
    and let $X^1 \defeq q(\geom{K^{\leq 1}})$. Clearly, since $K^\ast$ was a
    spanning tree, all vertices of~$K$ are identified in~$X^1$ and therefore
    \[ X^1 \cong \bigvee\nolimits^k S^1  , \]
    where $k\in\N$ is the number of $1$-simplices of~$K$ not contained
    in~$K^\ast$. For $j\in\setOneto k$ let $\gamma_j\colon S^1\to X^1$
    be the inclusion of the $j$-th loop into this wedge sum and let
    $[\eta_j]_\ast \defeq \psi([\gamma_j]_\ast)$. Then the loops
    $\eta_j\colon (S^1,[0])\to(Y,y_0)$ combine to a map
    \[ g_1\colon X^1 \to Y \]
    (because \enquote{$\vee$} is the coproduct in the category of pointed
    spaces). Now we extend $g_1$ to~$X$. Since only parts of
    $\geom{K^{\leq 1}}$ are identified to a point by~$q$, we see that
    $X$ fits into a pushout diagram
    \[
        \xymatrix@R=0.5cm@C=1.2cm{
            \coprod_{j=1}^\ell S^1 \ar[r]^-{\coprod_{j=1}^\ell \phi_j} \ar@{ `->}[d]
            & X^1 \ar[d]
            \\
            \coprod_{j=1}^\ell D^2 \ar[r]
            & X
            \rlap{,}
        }
    \]
    where $\ell\in\N$ is the number of $2$-simplices of~$K$. Fix
    $j\in\setOneto{\ell}$. Then
    \[
        [\phi_j]_\ast = [\gamma]_\ast \cdot [\gamma']_\ast \cdot [\gamma'']_\ast
        \;\in\,\pi_1(X,[x_0])
    \]
    where $\gamma,\gamma',\gamma''$ correspond to the three $1$-dimensional
    faces of the original $2$-simplex of~$K$, so each of those is either
    the constant loop at~$[x_0]\in X$ or one of the loops~$\gamma_{j'}$.
    It is easy to see that $\phi_j$ is pointed contractible in~$X$ because
    it is the projected boundary loop of a $2$-simplex in~$\geom{K}$.
    Furthermore,
    \[ [g_1\after\gamma]_\ast = \psi([\gamma]_\ast) \]
    (and likewise for $\gamma'$ and $\gamma''$) clearly holds if $\gamma$
    is the constant loop and for $\gamma = \gamma_{j'}$ we have
    $[g_1\after\gamma_{j'}]_\ast = [\eta_{j'}]_\ast = \psi([\gamma_{j'}]_\ast)$
    by definition of $g_1$ and $\eta_{j'}$.
    Now the following calculation shows that $g_1\after\phi_j$ is pointed
    contractible in~$Y$ as well:
    \begin{align*}
        [g_1\after\phi_j]_\ast
        &= [g_1\after\gamma]_\ast \cdot [g_1\after\gamma']_\ast \cdot
            [g_1\after\gamma'']_\ast
        \\
        &= \psi([\gamma]_\ast) \cdot \psi([\gamma']_\ast) \cdot
            \psi([\gamma'']_\ast)
        \\
        &= \psi([\phi_j]_\ast) = \psi(1) = 1 \in \pi_1(Y,y_0)
    . \end{align*}
    Thus $g_1\after\phi_j\colon S^1\to Y$ can be extended to a map
    $g_{2j}\colon D^2\to Y$ for all $j\in\setOneto\ell$.
    %
    We obtain a commutative diagram
    \[
        \xymatrix@R=0.5cm@C=1.2cm{
            \coprod_{j=1}^\ell S^1 \ar[r]^-{\coprod_{j=1}^\ell \phi_j} \ar@{ `->}[d]
            & X^1 \ar[d] \ar@/^1pc/[rdd]^{g_1}
            \\
            \coprod_{j=1}^\ell D^2 \ar[r] \ar@/_1pc/[rrd]_{\coprod_{j=1}^\ell g_{2j}}
            & X \ar@{-->}[rd]^g
            \\
            & & Y
            \rlap{,}
        }
    \]
    and by virtue of the universal property of the pushout a map
    $g\colon X\to Y$ fitting into the diagram. The composition
    \[ f\defeq g\after q\colon \geom K\to Y \]
    is the desired map because: the equivalence classes of the loops
    $\gamma_1,\dots,\gamma_k$ obviously generate~$\pi_1(X,[x_0])$ and $g$
    restricts to $g_1$ on~$X^1$, so $\pi(g) = \psi$; by functoriality of~$\pi_1$
    and definition of $\psi$, it follows that $\pi_1(f) = \phi$.
    \\
\end{proof}

\begin{thCorollary}[algebraic signature map vs. continous map]
    \label{ch3:algsignvscontinuous}
    %
    Let $K,L\in\finSimp$ and let $\kappa,\lambda$ be triangle loops
    in $K$ and $L$, respectively. Then there exists a map of
    algebraic signatures
    \[ \phi\colon \bigl( \pi_1(K,\dot\kappa), \, [\gamma_\kappa]_\ast \bigr)
        \to \bigl( \pi_1(L,\dot\lambda), \, [\gamma_\lambda]_\ast \bigr)
    \]
    if and only if there exists a continuous map
    \[ f\colon \geom K\to \geom L
        \qtextq{with} f\after\gamma_\kappa = \gamma_\lambda
    . \]
\end{thCorollary}

\begin{proof}
    If $f$ is a map as above, the induced homomorphism~$\pi_1(f)$ yields
    the map of algebraic signatures as required. Conversely, let $\phi$
    be such a map of algebraic signatures. By \cref{ch3:continuousrealization}
    there exists a map $f'\colon\geom K\to\geom L$ with $\pi_1(f')=\phi$.
    Then the loop~$f'\after\gamma_\kappa$ is, by assumption, pointed homotopic
    to~$\gamma_\lambda$; let the map $h\colon S^1\times[0,1]\to\geom L$ be such
    a pointed homotopy. By definition of~$\gamma_\kappa$ we see, that
    $\gamma_\kappa$ defines a homeomorphism between $S^1$ and a
    ($1$-dimensional) subcomplex of~$\geom L$, hence $\gamma_\kappa$
    is a (pointed) cofibration \pcref{ch1:inclcofibration}.
    Therefore, we obtain a pointed homotopy
    $H\colon\geom K\times[0,1]\to\geom L$ such that the following diagram
    commutes:
    \[
        \xymatrix@C=2cm{
            S^1
                \ar[r]^{\gamma_\kappa}
                \ar[d]_{(\id_{S^1},\,\const_0)}
            & \geom K
                \ar[d]_{(\id_{\geom K},\,\const_0)}
                \ar@/^1pc/[rdd]^{f'}
            \\
            S^1\times[0,1]
                \ar[r]_{\gamma_\kappa\times\id_{[0,1]}}
                \ar@/_1pc/[rrd]_h
            & \geom{K}\times[0,1]
                \ar@{-->}[rd]^H
            \\
            & & \geom L
            \rlap{.}
        }
    \]
    Let $f \defeq H(\scdot,1)\colon \geom K\to \geom L$, then
    \[ f\after\gamma_\kappa = H \after (\gamma_\kappa\times\const_1)
        = h(\scdot,1) = \gamma_\lambda
    \]
    and since $f$ and $f'$ are homotopic: $\pi_1(f) = \pi_1(f') = \phi$.
    \\
\end{proof}


\section{Classification and Consequences}
\label{ch2:sec:consequences}
%
\begingroup\newcommand{\mycite}{\cite[Theorem~15.3.8]{bookc:herlihyetal13}}
\begin{thTheorem}[classification of loop agreement tasks%
                  \texorpdfstring{ \protect\mycite}{}]
    \label{ch3:classification}
    %
    Let $K,L\in\finSimp$ and let $\kappa,\lambda$ be triangle loops
    in $K$ and $L$, respectively.
    Then $\Loop{K,\kappa}$ implements $\Loop{L,\lambda}$ if and only if
    there exists a map of algebraic signatures
    $( \pi_1(K,\dot\kappa), \, [\gamma_\kappa]_\ast )
        \to ( \pi_1(L,\dot\lambda), \, [\gamma_\lambda]_\ast )$.
\end{thTheorem}
\endgroup

Note that loop agreement tasks always have strict carrier maps (which
is immediate from the definition) and so the theorem makes sense as stated.

\begin{proof}[Proof of the theorem]
    \newcommand{\sdNkappa}{\sd^N\!\kappa}
    %
    For the whole proof let $\ppp = \kappa$ and $(q_0,q_1,q_2) = \lambda$.
    First, let a map of algebraic signatures as above exist. Then,
    by \cref{ch3:algsignvscontinuous}, there exists a map
    $f_\geo\colon\geom K\to\geom L$ such that
    $f_\geo\after\gamma_\kappa = \gamma_\lambda$. By the simplicial
    approximation theorem \pcref{ch1:simpapprox}, there exists an $N\in\N$ and a
    simplicial map $f\colon\sdNK \to L$ such that $f$ is a simplicial
    approximation to $f_\geo\after\beta_K^N$.
    We claim that $\sd_K^N\after\Loop{K,\kappa}$
    solves $\Loop{L,\lambda}$ with decision map~$f$,
    that is we have to show that $f\after(\sd_K^N\after\Loop{K,\kappa})$
    is carried by~$\Loop{L,\lambda}$.
    \[ \tag{$\star$} \label{ch3:classification:proof:star}
        \begin{gathered}
            \xymatrix@C=1.5cm{
                K \ar[r]^{\sd_K^N} & \sdNK \ar[d]^-f
                \\
                \Delta^2 \ar[u]^-{\Loop{K,\kappa}} \ar[r]^{\Loop{L,\lambda}} & L
            }
        \end{gathered}
    \]
    In the following, we identify the distinguished vertices of~$\kappa$ with
    the corresponding vertices of~$\sdNK$. Since $f$ is a simplicial
    approximation to $f_\geo\after\beta_K^N$ and $f_\geo(\geom{\dot{p_j}}) =
    \geom{\dot{q_j}}$ for all $j\in\{0,1,2\}$ by assumption on~$f_\geo$, we have
    \[ \supp\bigl(\geom{f}(\geom{\dot{p}_j})\bigr)
        \subset \supp\bigl( (f_\geo\after\beta_K^N)(\geom{\dot{p}_j}) \bigr)
        = \supp(\geom{\dot{q}_j}) = \{\dot{q}_j\}
    . \]
    The case of the $2$-simplex is trivial, so it remains to show
    that
    \[ \bigl(f\after(\sd_K^N\after\Loop{K,\kappa})\bigr)(F)
        \subset \Loop{L,\lambda}(F)
    \]
    for all $1$-simplices~$F$ of~$\Delta^2$. Let $F\in\Delta^2$ be a $1$-simplex
    and let $j\in\{0,1,2\}$ such that $\Loop{K,\kappa}(F) = K(p_j)$. Then, if
    $F'$ is a simplex of $K(p_j)$ and $F''$ is a simplex of $\sd_K^N(F')$, we must have
    $f(F'')\in L(q_j)$ because
    \[ (f_\geo\after\beta_K^N)(\geom{F''}) \subset f_\geo(\geom{F'})
        \subset f_\geo\bigl(\geom{K(p_j)}\bigr) \subset \geom{L(q_j)}
    \]
    and $f(F'')\notin L(q_j)$ would contradict the fact that $f$ is
    a simplicial approximation to $f_\geo\after\beta_K^N$. Therefore
    \[ \bigl(f\after(\sd_K^N\after\Loop{K,\kappa})\bigr)(F)
        = f\Bigl[ \bigcup \sd_K^N\bigl(K(p_j)\bigr) \Bigr]
        \subset L(q_j)
    , \]
    which completes the first half of the proof.
    
\pagebreak[2]
    Now assume that $\Loop{K,\kappa}$ implements $\Loop{L,\lambda}$ and that we
    are given $N\in\N$ and a simplicial map $f\colon\sdNK\to L$ such that
    $\sd_K^N\after\Loop{K,\kappa}$ solves $\Loop{L,\lambda}$ with decision
    map~$f$ (see \eqref{ch3:classification:proof:star}). Then we let
    \[ f_\geo \defeq \geom{f} \after (\beta_K^N)^{-1}
        \colon\geom K\to\geom L
    \]
    and claim that $\pi_1(f_\geo)$ is a map of algebraic signatures as desired.
    To that end, let $\gamma_{\sdNkappa}$ denote the loop
    $S^1\to\geom{\sdNK}$ that is defined like $\gamma_\kappa$
    \pcref{ch3:def:assoctopoloop} with the barycenters added to the cycle's
    vertices in the only sensible way. Then the following diagram commutes
    \[
        \xymatrix{
            S^1 \ar[r]^{\gamma_\kappa} \ar[dr]_{\gamma_{\sdNkappa}}
            & \geom{K}
            \\
            & \geom{\sdNK} \ar[u]_{\beta_K^N}
        }
    \]
    (which can be seen by inspection of the definitions).
    Furthermore, since the carrier map
    $f\after(\sd_K^N\after\Loop{K,\kappa})$ is
    carried by~$\Loop{L,\lambda}$, every simplex of~$\sd_K^N(K(p_j))$
    is mapped into~$L(q_j)$ by~$f$ and likewise for the corresponding
    geometric simplices. It follows that
    \[ \geom{f}\bigl(\geom[\big]{\sd_K^N[K(p_j)]\mkern1mu}\bigr)
        \subset \geom{L(q_j)}
    , \]
    which clearly implies that $\geom{f}\after\gamma_{\sdNkappa}$
    wraps around $\im(\gamma_\lambda)\cong S^1$ exactly once.
    Thus we have
    \[ f_\geo\after\gamma_\kappa 
        = \geom{f}\after(\beta_K^N)^{-1} \after\gamma_\kappa
        = \geom{f}\after\gamma_{\sdNkappa}
        \simeq_\ast \gamma_\lambda
    , \]
    which completes the proof.
    \\
\end{proof}

\pagebreak[2]
Now we list some immediate consequences of the classification theorem.
By the notation $(G,g) \cong (H,h)$ we always mean that there exists an group
isomorphism $G\to H$ with $g\mapsto h$.

\begin{thCorollary}[$(3,2)$-set agreement implements any loop agreement task]
    \label{ch3:32setagreementisboss}
    %
    Recall from \cref{ch3:32setagreement}, that $(3,2)$-set agreement can be
    interpreted as the loop agreement task~$\Loop{\partial\Delta^2,\,\kappa}$
    where $\kappa = ((0,1),\,(1,2),\,(2,0))$. Therefore the algebraic signature
    of~$\Loop{\partial\Delta^2,\,\kappa}$ is
    \[ \bigl( \pi_1(\partial\Delta^2,0), [\gamma_\kappa]_\ast) \bigr)
        \cong \bigl( \pi_1(S^1,[0]), [\id_{S^1}]_\ast \bigr)
        \cong (\Z,1)
    . \]
    For a group~$G$ and an element $g\in G$ there is exactly one group
    homomorphism $\Z\to G$ that maps $1$ to $g$, which implies, by the
    classification theorem~\pcref{ch3:classification}, that the $(3,2)$-set
    agreement task~$\Loop{\partial\Delta^2,\kappa}$ implements any other
    loop agreement task.
\end{thCorollary}

\begin{thCorollary}[no $(3,2)$-set agreement from barycentric agreement]
    Let $\Loop{\partial\Delta^2,\kappa}$ be the $(3,2)$-set agreement task as
    in the previous corollary, let $N\in\N$ and recall
    from~\cref{ch3:2dimbaryagreement} that $2$-dimensional $N$-th barycentric
    agreement (on $\Delta^2$) is a loop agreement task which we denote by
    $\Loop{\sdNK,\,\mu}$ here. Then $\Loop{\sdNK,\,\mu}$ does \emph{not}
    implement $\Loop{\partial\Delta^2,\kappa}$ because: $\sdNK$ is
    contractible, hence the algebraic signature of~$\Loop{\sdNK,\,\mu}$ is
    \[ \bigl( \pi_1(\sdNK,\dot\mu), [\const_{\geom{\dot\mu}}]_\ast
        \bigr) \cong (\{0\}, 0)
    \]
    and we already saw in \cref{ch3:32setagreementisboss} that the algebraic
    signature of $\Loop{\partial\Delta^2,\kappa}$ is (isomorphic to) $(\Z,1)$.
    Since there is only the trivial homomorphism $0\to\Z$, our assertion follows
    from the classification theorem~\pcref{ch3:classification}.
\end{thCorollary}
