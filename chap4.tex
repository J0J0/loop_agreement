\chapter{Two-dimensional Pseudomanifolds}
\section{Pseudomanifolds}
\begin{thDef}[(weak) pseudomanifold]
    \label{ch4:def:pseudomanifold}
    Let $n\in\N$. 
    \begin{itemize}
        \item
            An \emph{$n$-dimensional weak pseudomanifold (without boundary)}
            (or \emph{weak $n$-pseudomanifold}) is a simplicial complex~$K$
            of dimension~$n$ satisfying the following properties:
            \begin{itemize}
                \item
                    Every simplex of~$K$ is a face of an $n$-simplex of~$K$.
                \item
                    Every $(n{-}1)$-simplex of~$K$ is the face of exactly two
                    $n$-simplices of~$K$.
            \end{itemize}
            
        \item
            An \emph{$n$-dimensional pseudomanifold (without boundary)} (or
            \emph{$n$-pseudomanifold}) is a weak $n$-pseudomanifold~$K$ that is
            \emph{strongly connected}, that is it satisfies the following
            property:
            \begin{itemize}
                \item
                    For all $n$-simplices $F,F'$ of~$K$ there exists a $k\in\N$ and a
                    finite sequence $F=F_0,F_1,\dots,F_k=F'$ of $n$-simplices of~$K$
                    such that  $F_j\cap F_{j+1}$ is an $(n{-}1)$-simplex of~$K$ 
                    for all $j\in\setZeroto{k-1}$.
            \end{itemize}
            
        % TODO:  v clear!?
        %\item
        %    A (weak) $n$-pseudomanifold is \emph{finite} if $K$ is finite.
    \end{itemize}
\end{thDef}

Suppose $K$ is a weak $n$-pseudomanifold and let $x\in\geom K$.
Then the following fact is obvious from the definition:
if $\supp(x)$ is an $n$- or $(n{-}1)$-simplex of~$K$,
we can find a neighborhood of~$x$ that is homeomorphic
to~$\R^n$. On the other hand, if the dimension of~$\supp(x)$
is less than~$n{-}1$, singularities can occur at~$x$.
% TODO: add example
Thus, the geometric realization of~$K$ partially behaves like a manifold but may
fail to be locally euclidean at some points, hence the name \emph{pseudo}manifold.

\begin{thRemark}[(weak) pseudomanifolds with boundary and manifold protocols]
    If we replace the second property in \cref{ch4:def:pseudomanifold} by
    \begin{itemize}[label=\textasteriskcentered]
        \item
            Every $(n{-}1)$-simplex of~$K$ is the face of exactly one
            or two $n$-simplices of~$K$,
    \end{itemize}
    we get pseudomanifolds \emph{with boundary}. In this thesis we are only
    concerned with loop agreement tasks on $2$-dimensional pseudomanifolds
    without boundary, but the notion of pseudomanifolds with boundary plays
    an important role in so called \emph{manifold protocols}, studied by
    Herlihy et~al. in Chapter~9 of their book~\cite{bookc:herlihyetal13}.
    Briefly, a protocol~$\Xi\colon\cI\to\cP$ is a manifold protocol if the
    complex~$\Xi(F)\subset\cP$ is a pseudomanifold with boundary for
    all~$F\in\cI$ (and $\Xi$ commutes with taking boundaries in an appropriate
    way). For instance, it is easy to see that the barycentric agreement
    protocol~$\sd_K$ \pcref{ch2:barycentricagreement} is a manifold protocol
    for all~$K\in\Simp$.
\end{thRemark}


\section{Classification of 2-dimensional Pseudomanifolds}

\section{Implementation}

\section{Loop Agreement Tasks on 2-dimensional Pseudomanifolds}
